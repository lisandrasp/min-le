\documentclass{beamer}

\usepackage[portuguese]{babel}
\usepackage[utf8]{inputenc}

\usepackage{amsmath}
\usepackage{tikz}
\usepackage{minted}
\usepackage{pgfplots}
\usepackage{nicematrix}

\pgfplotsset{width=8cm,compat=1.17}

\title{\huge Grafos unicíclicos com energia laplaciana mínima}
\author{\Large Lisandra Simões Pires \and \text{\normalsize Orientadora: Prof. Dra. Virgínia Maria Rodrigues}}
\institute{\includegraphics[height=1cm]{Imagens/ime-logotipo.png}}
\date{\tiny Setembro 2021}
\logo{\includegraphics[height=1cm]{Imagens/salao-2021-preto-vermelho.png}}

\begin{document}

\frame{\titlepage}

\begin{frame}{Grafo unicíclico}
    \begin{figure}[h]
        \begin{tikzpicture}[every node/.style={draw, circle, inner sep=0pt}]
    \node (1) {$v_1$};
    \node (2) [above right of=1] {$v_2$};
    \node (3) [below right of=1] {$v_3$};
    \node (4) [below right of=2] {$v_4$};
    \node (5) [right of=4] {$v_5$};
    \node (6) [right of=5] {$v_6$};
    \node (7) [right of=6] {$v_7$};
    \draw (1) -- (2);
    \draw (1) -- (3);
    \draw (2) -- (4);
    \draw (3) -- (4);
    \draw (4) -- (5);
    \draw (5) -- (6);
    \draw (6) -- (7);
\end{tikzpicture}
        \caption{O grafo $T_{4,3}$.}
    \end{figure}
    
    \begin{itemize}
        \item Um grafo é uma estrutura $G=G(V, E)$, constituída por um conjunto $V$, cujos elementos são denominados vértices, e um conjunto $E$ de subconjuntos formados por pares não ordenados de $V$.
        \item Um \alert{grafo unicíclico} é um \alert{grafo conexo} contendo exatamente um ciclo.
        \item O grau de um vértice $v$, denotado por $d(v)$, é o número de vértices adjacentes a $v$.
    \end{itemize}
\end{frame}

\begin{frame}{Matriz laplaciana}
    A \alert{matriz laplaciana} é a matriz $L_{nxn}$ com entradas
    \[L_{i,j}:=
    \begin{cases}
        \text{$d(v_i)$} &\quad\text{se $i=j$} \\
        \text{$-1$} &\quad\text{se $i \neq j$ e $v_i$ é adjacente a $v_j$} \\
        \text{$0$} &\quad\text{caso contrário} \\
    \end{cases}\]
    
    \begin{columns}
    \column{0.5\textwidth}
    \begin{figure}[h]
        \resizebox{5cm}{!}{\begin{tikzpicture}[every node/.style={draw, circle, inner sep=0pt}]
    \node (1) {$v_1$};
    \node (2) [above right of=1] {$v_2$};
    \node (3) [below right of=1] {$v_3$};
    \node (4) [below right of=2] {$v_4$};
    \node (5) [right of=4] {$v_5$};
    \node (6) [right of=5] {$v_6$};
    \node (7) [right of=6] {$v_7$};
    \draw (1) -- (2);
    \draw (1) -- (3);
    \draw (2) -- (4);
    \draw (3) -- (4);
    \draw (4) -- (5);
    \draw (5) -- (6);
    \draw (6) -- (7);
\end{tikzpicture}}
    \end{figure}
    
    \column{0.5\textwidth}
    \begin{figure}[h]
        \resizebox{5cm}{!}{$\begin{bNiceMatrix}[first-row,first-col]
    & v_1 & v_2 & v_3 & v_4 & v_5 & v_6 & v_7\\
    v_1 & 2 & -1 & -1 & 0 & 0 & 0 & 0 \\
    v_2 & -1 & 2 & 0 & -1 & 0 & 0 & 0 \\
    v_3 & -1 & 0 & 2 & -1 & 0 & 0 & 0 \\
    v_4 & 0 & -1 & -1 & 3 & -1 & 0 & 0 \\
    v_5 & 0 & 0 & 0 & -1 & 2 & -1 & 0 \\
    v_6 & 0 & 0 & 0 & 0 & -1 & 2 & -1 \\
    v_7 & 0 & 0 & 0 & 0 & 0 & -1 & 1
\end{bNiceMatrix}$}
    \end{figure}
    \end{columns}
\end{frame}

\begin{frame}{Energia laplaciana}
    A \alert{energia laplaciana} \cite{gutman} de um grafo $G$ de ordem $n$ com $m$ arestas é definida como \[LE(G)=\sum_{i=1}^{n}|\mu_{i}-\bar{d}|\] onde $\mu_{1} \geq \mu_{2} \geq ... \geq \mu_{n}$ são os \textcolor{blue}{autovalores laplacianos} de $G$ e $\bar{d}=\frac{2m}{n}$ é a média dos graus.
    
    \begin{equation*}
        \begin{split}
            LE(T_{4,3}) & =|\textcolor{blue}{0}-2|+|\textcolor{blue}{0,27652}-2|+|\textcolor{blue}{2}-2|+|\textcolor{blue}{1,33230}-2| \\ & +|\textcolor{blue}{2,52195}-2|+|\textcolor{blue}{3,29200}-2|+|\textcolor{blue}{4,57723}-2| \\ & =8,78236
        \end{split}
    \end{equation*}
\end{frame}

\begin{frame}{Grafos extremais com respeito à energia laplaciana}
    \begin{table}[h]
        \resizebox{10cm}{!}{\begin{tabular}{|c|c|c|}
    \hline
    & Árvore & Unicíclico \\
    \hline
    \rule{0pt}{12ex}
    Maior energia & \resizebox{2cm}{!}{
    \centering
    \begin{tikzpicture}[every node/.style={fill, circle, inner sep=1.5pt}]
        \node (1) {};
        \node (2) [left of=1] {};
        \node (3) [above left of=1] {};
        \node (4) [above of=1] {};
        \node (5) [above right of=1] {};
        \node (6) [right of=1] {};
        \node (7) [below right of=1]{};
        \node (8) [below of=1] {};
        \node (9) [below left of=1] {};
        \draw (1) -- (2);
        \draw (1) -- (3);
        \draw (1) -- (4);
        \draw (1) -- (5);
        \draw (1) -- (6);
        \draw (1) -- (7);
        \draw (1) -- (8);
        \draw (1) -- (9);
    \end{tikzpicture}
} & \resizebox{2.5cm}{!}{
    \centering
    \begin{tikzpicture}[every node/.style={fill, circle, inner sep=1.5pt}]
        \node (1) {};
        \node (2) [below left of=1] {};
        \node (3) [below right of=1] {};
        \node (4) [above left of=2] {};
        \node (5) [below left of=2] {};
        \node (6) [above right of=3] {};
        \node (7) [above left of=1] {};
        \node (8) [above right of=1] {};
        \node (9) [below right of=3] {};
        \draw (1) -- (2);
        \draw (1) -- (3);
        \draw (1) -- (7);
        \draw (1) -- (8);
        \draw (2) -- (3);
        \draw (2) -- (4);
        \draw (2) -- (5);
        \draw (3) -- (6);
        \draw (3) -- (9);
    \end{tikzpicture}
} \\
    & $S_n$ & $H_n$ \\
    & Provado & Conjectura provada parcialmente \\
    \hline
    \rule{0pt}{6ex}
    Menor energia & \resizebox{2.5cm}{!}{
    \centering
    \begin{tikzpicture}[every node/.style={fill, circle, inner sep=2pt}]
        \node (1) {};
        \node (2) [right of=1] {};
        \node (3) [right of=2] {};
        \node (4) [right of=3] {};
        \node (5) [right of=4] {};
        \node (6) [below of=5] {};
        \node (7) [left of=6]{};
        \node (8) [left of=7] {};
        \node (9) [left of=8] {};
        \draw (1) -- (2);
        \draw (2) -- (3);
        \draw (3) -- (4);
        \draw (4) -- (5);
        \draw (5) -- (6);
        \draw (6) -- (7);
        \draw (7) -- (8);
        \draw (8) -- (9);
    \end{tikzpicture}
} & ? \\
    & $P_n$ & \\
    & Conjectura em aberto & \\
    \hline
\end{tabular}}
    \end{table}
\end{frame}

\begin{frame}{Problema}
Dentre todos os grafos unicíclicos com $n$ vértices, qual é o de menor energia laplaciana?
\end{frame}

\begin{frame}{Ferramentas}
\begin{description}
    \item[Python] Linguagem de programação
    \item[SageMath] Software de matemática de código aberto 
    \item[Nauty] Programas para calcular grupos de automorfismos de grafos e digrafos
\end{description}
\end{frame}

\begin{frame}[fragile]{Código}
    \begin{minted}[
frame=lines,
framesep=2mm,
bgcolor=lightgray,
fontsize=\tiny
]
{python}

    while f - 240 <= len(tuple_nauty) + 1:
        for new_graph_tuple in tuple_nauty[s:f]:
            new_spectrum = new_graph_tuple[1].spectrum(laplacian=True)
            new_energy = laplacian_energy(new_spectrum, n, m)
            new_index = new_graph_tuple[0]

            if new_energy < energy:
                spectrum = new_spectrum
                energy = new_energy
                graph_tuple = new_graph_tuple
                index = new_index
                graphs_nauty[index].plot().save(str('graph_partial_' + str(n) + '.png'))
        s += 240
        f += 240
        cycle += 1
        print(f'Cycle {cycle} Partial minimum Laplacian energy {round(energy, 5)}')

    degree = graph_tuple[1].average_degree()
    sigma = len([mu for mu in spectrum if mu >= degree])
    diameter = graph_tuple[1].diameter()
    graphs_nauty[index].plot().save(str('graph_' + str(n) + '.png'))
\end{minted}
    \textcolor{red}{Códigos disponíveis em \url{https://github.com/lisandrasp}}
\end{frame}

\begin{frame}{Crescimento exponencial}
    \begin{table}[h]
        \begin{tabular}{|c|c|}
    \hline
    \# grafos & Tempo \\
    \hline
    2 & 0.28 s \\
    \hline
    5 & 0.3 s \\
    \hline
    13 & 0.48 s \\
    \hline
    33 & 0.74 s \\
    \hline
    89 & 1.41 s \\
    \hline
    240 & 3.01 s \\
    \hline
    657 & 7.89 s \\
    \hline
    1806 & 22.69 s \\
    \hline
    5026 & 63.09 s \\
    \hline
    13999 & 221.42 s \\
    \hline
    39260 & 681.17 s \\
    \hline
    110381 & 2173.25 s \\
    \hline
    311465 & 6361.53 s \\
    \hline
    880840 & 19705.16 s \\
    \hline
\end{tabular}
        \caption{Número de grafos e tempo de processamento.}
    \end{table}
\end{frame}

\begin{frame}{Resultados}
    \begin{table}[h]
        \begin{tabular}{|c|c|c|c|c|c|c|}
    \hline
    $n$ & $LE$ & Unicíclico $LE$ min. \\
    \hline
    4 & 4.0 & $T_{4,0}$ \\
    \hline
    5 & 6.34017 & $T_{4,1}$ \\
    \hline
    6 & 7.12311 & $T_{4,2}$ \\
    \hline
    7 & 8.78236 & $T_{4,3}$ \\
    \hline
    \textcolor{red}{8} & \textcolor{red}{9.65685} & \textcolor{red}{$C_{8}$} \\
    \hline
    9 & 11.2951 & $T_{4,5}$ \\
    \hline
    10 & 12.35994 & $T_{4,6}$ \\
    \hline
    11 & 13.82402 & $T_{4,7}$ \\
    \hline
    \textcolor{red}{12} & \textcolor{red}{14.9282} & \textcolor{red}{$C_{12}$} \\
    \hline
    13 & 16.35959 & $T_{4,9}$ \\
    \hline
    14 & 17.49738 & $T_{4,10}$ \\
    \hline
    15 & 18.89861 & $T_{4,11}$ \\
    \hline
    16 & 20.05644 & $T_{4,12}$ \\
    \hline
    17 & 21.43965 & $T_{4,13}$ \\
    \hline
\end{tabular}
        \caption{Resultados para grafos unicíclicos onde $4 \leq n \leq 17$.}
    \end{table}
\end{frame}

\begin{frame}{Resultados}
    \begin{table}[h]
        \begin{tabular}{|c|c|c|}
\hline
$n$ & Uni. $LE$ min. & Uni. 2º $LE$ min. \\
\hline
8 & $C_{8}$ & $T_{4,4}$ \\
\hline
12 & $C_{12}$ & $T_{4,8}$ \\
\hline
\end{tabular}
        \caption{Casos especiais onde o grafo com menor energia não é um \emph{tadpole}.}
    \end{table}
\end{frame}

\begin{frame}{Resultados parciais}
    \begin{table}[h]
        \begin{tabular}{|c|c|c|c|c|}
    \hline
    $n$ & \# grafos & \% grafos proc. & $LE$ parcial & $LE$ $T_{4,n-4}$ \\
    \hline
    18 & 2497405 & 31.74 \% & 26.84711 & 22.61233 \\
    \hline
    19 & 7093751 & 21.39 \% & 28.0331 & 23.98198 \\
    \hline
    20 & 20187313 & 5.85 \% & 30.79398 & 25.16611 \\
    \hline
\end{tabular}
        \caption{Comparação entre os resultados parciais e os grafos \emph{tadpole}.}
    \end{table}    
\end{frame}

\begin{frame}{Conjectura}
    Dentre todos os unicíclicos com $n$ vértices, o de menor energia laplaciana é o \emph{tadpole} $T_{4,n-4}$, exceto quando $n=8$ e $n=12$.
\end{frame}

\begin{frame}{Parâmetros}
    \begin{itemize}
        \item O \alert{$\sigma$} é o número de autovalores laplacianos de $G$ maiores ou iguais ao grau médio $\bar{d}$. \cite{pinheiro}
        \item O \alert{diâmetro} é a maior distância entre qualquer par de vértices.
    \end{itemize}
\end{frame}

\begin{frame}{Parâmetros}
    \begin{table}[h]
        \begin{tabular}{|c|c|c|}
    \hline
    $n$ & $\sigma$ & $diam$ \\
    \hline
    4 & 3 & 2 \\
    \hline
    5 & 3 & 3 \\
    \hline
    6 & 4 & 4 \\
    \hline
    7 & 4 & 5 \\
    \hline
    \textcolor{red}{8} & \textcolor{red}{5} & \textcolor{red}{4} \\
    \hline
    9 & 5 & 7 \\
    \hline
    10 & 6 & 8 \\
    \hline
    11 & 6 & 9 \\
    \hline
    \textcolor{red}{12} & \textcolor{red}{7} & \textcolor{red}{6} \\
    \hline
    13 & 7 & 11 \\
    \hline
    14 & 8 & 12 \\
    \hline
    15 & 8 & 13 \\
    \hline
    16 & 9 & 14 \\
    \hline
    17 & 9 & 15 \\
    \hline
\end{tabular}
        \caption{$\sigma$ e diâmetro para os grafos unicíclicos.}
    \end{table}    
\end{frame}

\begin{frame}{Parâmetros}
    \begin{alertblock}{$\sigma$}
        \[\sigma =
        \begin{cases}
            \text{$\frac{n}{2}+1$} &\quad\text{se $n$ é par}\\
            \text{$\frac{n+1}{2}+1$} &\quad\text{se $n$ é ímpar}\\
        \end{cases}\]
    \end{alertblock}
    
    \begin{alertblock}{Diâmetro}
        \[diam =
        \begin{cases}
            \text{$n-2$} &\quad\text{se $n \ne 8$ ou $n \ne  12$}\\
            \text{$\frac{n}{2}$} &\quad\text{se $n=8$ ou $n=12$}\\
        \end{cases}\]
    \end{alertblock}
\end{frame}

\begin{frame}
    \begin{thebibliography}{5}
        \setbeamertemplate{bibliography item}[text]
        \bibitem{gutman}
        GUTMAN, I.; ZHOU, B. Laplacian energy of a graph. Linear Algebra and its Applications, v. 414, n. 1, p. 29-37, abr. 2006.
        \bibitem{pinheiro}
        PINHEIRO, L. K. Energia laplaciana sem sinal de grafos. 2018. 104 p. Tese (doutorado) - Universidade Federal do Rio Grande do Sul, Porto Alegre, 2018
    \end{thebibliography}
\end{frame}

\end{document}
